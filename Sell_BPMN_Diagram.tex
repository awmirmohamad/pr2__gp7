\documentclass[12pt,onecolumn,a4paper]{article}
\usepackage{float}
\usepackage{epsfig,graphicx,subfigure,amsthm,amsmath}
\usepackage{color,xcolor}
\usepackage{xepersian}
\usepackage{graphicx}

\settextfont[Scale=1.2]{BZAR.TTF}
\setlatintextfont[Scale=1]{Times New Roman}


\begin{document}
\newpage
\tableofcontents
\newpage
\listoffigures
\newpage


\section{نمودار مدل\lr{BPMN} فرآیند خرید}
\subsection{توضیخات مدل }

در نمودار مدل ایجاد شده پیک را به دلیل اینکه از ابتدا تا انتها با پلتفرم درگیر است بعنوان \lr{lane}  در کنار بخش فروش سایت در یک \lr{pool} به نام وب‌سایت فروش در نظر گرفته شده و دیگر بخش ها همان‌‌طور که در نمودار مشخص است شامل: بانک، مشتری و فروشگاه بعنوان \lr{pool} است.
فرآیند پرداخت نیز در توالی مربوط به مشتری به دلیل مجزا بودن از کل فرآیند و به نوعی فرا بخشی بودن در یک \lr{Sub- process} شده.
به دلیل چالش‌ها و نقص‌ها نرم افزار \lr{Bizagi} که نه تنها امکان انتقال از درون یک \lr{Sub- process} وجود ندارد بلکه اصلاح و ایجاد تغییر در ان نیز به طور کامل امکان‌پذیر نبوده و بعضی بخش ها در تصویر نهایی نمودار ادیت شده.
توالی اطلاع سایت فروش و فروشگاه از ارسال سفارش منطقا اینگونه  فرض شده و فروشگاه تنها پس از دریافت رسید پرداخت محصول را تحویل پیک می‌دهد و به نوعی پیش‌نیازی تبدیل شده.
 برای اجتناب از رسم ارتباطات غیر ضروری میان مشتری و سایت فروش فرض شده که تکمیل کردن مراحل خرید اقدامی از سوی مشتری بوده و به دلیل عدم اطلاع دقیق از کنش و واکنش های سایت تنها در آخرین مرحله پس از پرداخت پیغامی به عنوان رسید و پرداخت و سفارش به وب‌سایت فروش انتقال داده می‌شود.
در آخر باید این نکته را متذکر شویم که این تنها روش مدلسازی این فرآیند نیست و می‌توان و با تغییر فرضیات و جیدمان \lr{Pool , lane}های متفاوت به نحوه دیگری نمودار را رسم کرد اما مهم وجود تمامی فعالیت ها با رعایت توالی آنهاست که در مدل رسم شده به این موضوعات توجه شده.


\subsection{پیشنهاد }
بنده بعد از این تجربه دیگر هرگز از نرم‌افزار \lr{Bizagi} به دلیل نقص‌های زیاد و جدی آن استفاده نخواهم کرد به همین دلیل بعنوان یک پیشنهاد لطفا در ترم‌های آینده نرم‌افزار دیگری جایگزین نمایید.

\newpage
\begin{figure}[!h]
\centering{\includegraphics[width=14cm]{Sell_Bpmn_Diagram}}
\caption{نمودار مدل\lr{BPMN} فرآیند خرید}\label{sellbpmnpng}
\end{figure}




\end{document} 